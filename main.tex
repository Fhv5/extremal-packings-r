\documentclass[12pt,letterpaper]{article}

% Paquetes básicos
\usepackage[utf8]{inputenc}
\usepackage[spanish]{babel}
\usepackage[letterpaper,margin=1in]{geometry}
\usepackage{graphicx}
\usepackage{amsmath}
\usepackage{amsthm}
\usepackage{amssymb}
\usepackage{hyperref}
\usepackage{xcolor}
\definecolor{pastelgreen}{rgb}{0.47, 0.87, 0.47} % Define pastelgreen color
\usepackage{tikz}
\usetikzlibrary{babel}       
\usetikzlibrary{arrows.meta, calc}  

\newtheorem{theorem}{Teorema}

% Información del documento
\title{Perímetros extremos en empaquetamientos de discos congruentes}
\author{Fabián Henry Vilaxa}
\date{\today}

\begin{document}

\maketitle

\begin{abstract}
    Mi investigación está enmarcada en el problema del perímetro minimo de la envolvente
    convexa de empaquetamientos de discos congruentes. Concretamente, busco resolver el siguiente problema:
    Dado un grafo de contacto conexo de discos congruentes en el plano, 
    determinar si dicho grafo constituye un punto crítico del perímetro de la envolvente convexa de su clase de contacto.
    De ser así, indicar si corresponde a un mínimo o máximo. En este documento describo y fundamento los elementos esenciales
    del procedimiento basado en el cálculo variacional que propongo para resolver este problema.
\end{abstract}

\section{El problema del perímetro de la envolvente convexa}

El problema central que motiva esta investigación es bastante visual. Supongamos que tenemos un montón de discos 
del mismo radio sobre una mesa y les ponemos una banda elástica alrededor. Lo que nos interesa es encontrar la disposición 
(o configuración) de estos discos que hace que esa banda elástica tenga la menor longitud posible. Es decir, 
buscamos minimizar el perímetro de la envolvente convexa de los discos.

En lugar de intentar buscar la mejor configuración entre las infinitas posibilidades (lo cual es computacionalmente infactible), nos enfocamos
en un problema distinto, pero en la misma línea.

No buscaremos el mínimo global directamente. En su lugar, tomaremos un grafo de contacto específico (una "forma" de conectar los discos) 
y utilizaremos herramientas de cálculo variacional para responder dos preguntas fundamentales sobre él:
\begin{enumerate}
    \item \textbf{Criticidad:} Si posicionamos los discos de una forma específica, ¿Será que esa dispoción entrega un punto crítico del perímetro? (Primera Variación).
    \item \textbf{Estabilidad:} Si es así, ¿Qué clase de estabilidad tiene ese punto crítico? (Segunda Variación).
\end{enumerate}

\section{Preliminares}
\subsection{Representación de empaquetamientos}
Los empaquetamientos los representamos mediante grafos de contacto.
Definimos a un grafo como una estructura $G$ que tiene aristas $E$ y vertices $V$.
Por ejemplo, el siguiente grafo tiene vertices $V = \{v_1, v_2, v_3\}$ y aristas $E = \{(v_1, v_2), (v_2, v_3), (v_3, v_1)\}$.

%Ejemplo de grafo normal triangular, con 3 vertices y 3 aristas.
\begin{center}
\begin{tikzpicture}[scale=1.5]
    % Vértices
    \node[circle, fill=black, inner sep=2pt, label=below left:$v_1$] (v1) at (-1,0) {};
    \node[circle, fill=black, inner sep=2pt, label=below right:$v_2$] (v2) at (1,0) {};
    \node[circle, fill=black, inner sep=2pt, label=above:$v_3$] (v3) at (0,1.732) {};
    
    % Aristas
    \draw[thick, orange] (v1) -- (v2);
    \draw[thick, orange] (v2) -- (v3);
    \draw[thick, orange] (v3) -- (v1);
\end{tikzpicture}
\end{center}


En un grafo de contacto, los vertices se ``expanden'' de manera que cada uno se transforme 
en un disco. En nuestro caso, concretamente en un disco de radio 1.
En cuanto a las aristas del grafo, se representan como un ``contacto'' entre dos discos, es decir,
los discos que representan los vertices conectados por una arista son tangentes entre sí.

% Ejemplo de grafo de contacto triangular, con 3 discos en contacto.
\begin{center}
\begin{tikzpicture}[scale=1.5]
    % Discos
    \draw[thick] (0,1.732) circle(1);
    \draw[thick] (-1,0) circle(1);
    \draw[thick] (1,0) circle(1);
    
    % Centros
    \node[circle, fill=black, inner sep=2pt] (c1) at (-1,0) {};
    \node[circle, fill=black, inner sep=2pt] (c2) at (1,0) {};
    \node[circle, fill=black, inner sep=2pt] (c3) at (0,1.732) {};
    
    % Aristas entre centros
    \draw[thick, dashed, orange] (c1) -- (c2);
    \draw[thick, dashed, orange] (c2) -- (c3);
    \draw[thick, dashed, orange] (c3) -- (c1);
    
    % Etiquetas cerca de los centros
    \node at (-1.3, -0.2) {$c_1$};
    \node at (1.3, -0.2) {$c_2$};
    \node at (0, 2) {$c_3$};
\end{tikzpicture}
\end{center}

Para nuestra investigacion, cada realización de uno de estos grafos en el plano
le llamamos \textbf{configuración}, y la representamos mediante un
vector $c$ de $\mathbb{R}^{2n}$, donde $n$ es el número de discos en la configuración. 
\begin{equation}
    c =(c_1, c_2, ..., c_n) \in \mathbb{R}^{2n}.
\end{equation}
Entonces, si tenemos 3 discos,
la configuración se representa como un vector 
\[
    c = (c_1, c_2, c_3),
\]

donde cada $c_i$ representa un par ordenado $(x_i, y_i)$ que indica la posición del 
centro del disco $c_i$ en el plano. En forma expandida:
\[
    c = (x_1, y_1, x_2, y_2, x_3, y_3).
\]

\subsection{Espacio de Configuraciones}
¿Con qué configuraciones exactamente estamos trabajando? ¿Que propiedades tienen?
En primer lugar, nuestras configuraciones:
\begin{itemize}
    \item Son \textbf{conexas}: el grafo de contacto asociado a la configuración es conexo.
    \item Son \textbf{no superpuestas}: ningún par de discos se intersecta
\end{itemize}

Que el grafo sea conexo significa que no puede existir un vertice o conjunto de vertices ``aislados'', 
o que no esté conectado al resto del grafo. Dicho de otra forma, todos los discos deben tener
al menos un contacto con otro disco.\\

Debido a que estamos trabajando con empaquetamientos, no tiene sentido considerar
que nuestros grafos puedan tener discos que se intersecten entre sí, por lo que nuestra restricción
principal es que todo centro del grafo debe estar a una distancia minima de $2$.
Es decir, la distancia del disco $c_j$ y $c_i$ debe ser $\geq2$, o $=2$ si existe contacto entre $c_j$ y $c_i$.
\\

Así, todas las configuraciones ``admisibles'' pueden definirse como un conjunto de grafos conexos dado por:

\begin{equation}
    \mathcal{D}_n = \{ c \in \mathbb{R}^{2n} \mid ||c_i - c_j|| \geq 2, \forall i \neq j \}
\end{equation}

En esta definición, $||c_i - c_j||$ denota la distancia euclidiana entre los centros de los discos $c_i$ y $c_j$, 
y $D_n$ es el espacio de configuraciones de $n$ discos congruentes en el plano.\\

Podemos descomponer el espacio $\mathcal{D}_n$ en conjuntos que contienen todos 
los grafos de contacto cuya estructura combinatoria es la misma; es decir, ``agrupar''
todos los grafos que tienen las mismas aristas y vertices, pero con diferentes posiciones en el plano.
A cada uno de estos conjuntos los llamamos \textbf{clases de contacto} o \textbf{estratos},
denotados como $\mathcal{C}(G)$. Podemos definir una clase de contacto como:
\begin{equation}
    \mathcal{C}(G) = \{ c \in \mathcal{D}_n: ||c_j - c_i|| = 2, \forall (i,j) \in E(G)\}
\end{equation}

\subsection{Bordes de la envolvente convexa}
Para calcular el perímetro de la envolvente de una configuración, es necesario
primero que determinemos qué es la envolvente convexa exactamente.
Consideramos que la envolvente son las lineas que forman el perímetro exterior del conjunto de discos
de la configuración. Concretamente, es la suma de las distancias entre los vertices que forman parte del perímetro exterior,
en orden ciclico, + 2$\pi$ (la longitud de los arcos de los discos que forman parte del perímetro).\\


Supongamos que el perímetro conecta una secuencia de discos identificados por los índices $v_1, v_2, \dots, v_h$. 
Definimos $\mathcal{B}$ como el conjunto de pares ordenados de índices que conforman este ciclo:

\begin{equation}
    \mathcal{B} = \{ (v_1, v_2), (v_2, v_3), \dots, (v_{h-1}, v_h), (v_h, v_1) \}.
\end{equation}

Cada elemento $(u, v) \in \mathcal{B}$ es un par de enteros que indica que existe una 
conexión dirigida desde el disco con índice $u$ hacia el disco con índice $v$. \\

Consideremos a modo de ejemplo una configuración $c = \{c_1, c_2, c_3\}$, con contactos
$E = \{(c_1,c_2),(c_2,c_3),(c_3,c_1)\}$, y centros ubicados en:
\[
    c_1 = (0, 0), \quad c_2 = (2, 0), \quad c_3 = (1, \sqrt{3}).
\]

\begin{center}
    \begin{tikzpicture}[scale=1.5]
        % Discos
        \draw[thick, black] (-1,0) circle(1);
        \draw[thick, black] (1,0) circle(1);
        \draw[thick, black] (0,1.732) circle(1);
        
        % Centros
        \node[circle, fill=black, inner sep=2pt] (c1) at (-1,0) {};
        \node[circle, fill=black, inner sep=2pt] (c2) at (1,0) {};
        \node[circle, fill=black, inner sep=2pt] (c3) at (0,1.732) {};
        
        % Aristas entre centros
        \draw[thick, dashed, orange] (c1) -- (c2);
        \draw[thick, dashed, orange] (c2) -- (c3);
        \draw[thick, dashed, orange] (c3) -- (c1);
        
        % Etiquetas cerca de los centros
        \node at (-1.3, -0.2) {$c_1$};
        \node at (1.3, -0.2) {$c_2$};
        \node at (0, 2) {$c_3$};
        
        % Envolvente convexa - arcos externos (120 grados cada uno)
        % Arco superior del disco c1
        \draw[thick, cyan] (0.866, 2.232) arc (30:150:1);
        % Arco izquierdo del disco c2
        \draw[thick, cyan] (-1.866, 0.5) arc (150:270:1);
        % Arco derecho del disco c3
        \draw[thick, cyan] (1.866, 0.5) arc (30:-90:1);
        
        % Tangentes externas comunes
        % Tangente de c1 a c2
        \draw[thick, cyan] (-0.866, 2.232) -- (-1.866, 0.5);
        % Tangente de c2 a c3
        \draw[thick, cyan] (-1, -1) -- (1, -1);
        % Tangente de c3 a c1
        \draw[thick, cyan] (1.866, 0.5) -- (0.866, 2.232);
    \end{tikzpicture}
    \end{center}

La envolvente convexa está dada por $\mathcal{B} = \{(1, 2), (2, 3), (3, 1)\}$.
Podemos diferenciar entre la envolvente de los centros y la envolvente de los discos.
Para efectos prácticos son esencialmente lo mismo, pero en casos como la cadena colineal
es importante hacer la distinción.

\subsection{Funcional Perímetro}
Con todo esto en mente, definimos el funcional perímetro como la suma de las aristas
que conforman la envolvente convexa de los centros más la longitud de los arcos:
\begin{equation}
    \text{Per}(c) = \sum_{(u,v) \in \mathcal{B}} ||c_v - c_u|| + 2\pi.
\end{equation}

Siguiendo nuestro ejemplo anterior ($\mathcal{B} = \{(1, 2), (2, 3), (3, 1)\}$), 
tenemos que el perímetro de la configuración $c = \{c_1, c_2, c_3\}$ es:
\[
    \text{Per}(c) = ||c_2 - c_1|| + ||c_3 - c_2|| + ||c_1 - c_3|| + 2\pi = 6 + 2\pi.
\]

\section{Primera Variación: Determinando criticidad}
Cuando buscamos obtener puntos críticos de una función, por lo general aplicamos
la derivada y buscamos los puntos donde esta sea hace cero o se indetermina. Como nuestra función
depende de multiples variables (los $2n$ valores que representan la configuración), utilizamos la
derivada direccional, que nos indica la tasa de cambio de la función en una dirección que nosotros
especifiquemos. Esa ``dirección'' la representamos mediante un vector $\delta c$, que en la práctica
representa una perturbación a la configuración tal que la forma en la que los discos estaban inicialmente
dispuestos cambie un poco.

Concretamente, buscamos configuraciones de $\mathcal{C}(G)$ tales que para cualquier variación admisible
$\delta c$ que mantenga la configuración dentro de la clase de contacto, la primera variación del perímetro sea cero:
\begin{equation}
    D \text{Per}(c)[\delta c] = 0.
    \label{eq:criticidad}
\end{equation}

\subsection{¿Qué es una variación admisible?}
Cuando hablamos de una variación o \textbf{movimiento} admisible, nos referimos a una perturbación que si bien modifica
la posición inicial de los discos, mantiene los contactos originales entre ellos. Dicho
de otra forma, la configuración resultante de aplicar la variación debe seguir perteneciendo
a la misma clase de contacto $\mathcal{C}(G)$.\\

Todo el conjunto de variaciones admisibles constituye un subespacio vectorial de $\mathbb{R}^{2n}$,
denotado como $T_c \mathcal{C}(G)$, y llamado el \textbf{espacio tangente} a la clase de contacto $\mathcal{C}(G)$
en el punto $c$. Este espacio representa todas las posibles direcciones en las que podemos movernos
desde la configuración $c$ sin salir de la clase de contacto.\\

Debido a que fisicamente estos movimientos se corresponden con la idea de ``rodar'' un
disco por sobre otro, en adelante denotaremos a este espacio $\text{Roll}(c)$. Para calcularlo,
utilizamos las restricciones de contacto sobre las que hablamos antes:
\[
    ||c_i - c_j|| = 2, \quad \forall (i,j) \in E(G).
\]
y las derivaremos. Esto nos da un sistema de ecuaciones lineales que podemos representar como una matriz.
El kernel de esta matriz es precisamente el espacio tangente.

\subsection{El gradiente y el espacio tangente}
El gradiente de una función es un vector que apunta en la dirección de mayor aumento de la función.
Convenientemente, el gradiente es una herramienta muy útil para saber ``hacia donde podemos movernos''
dentro de una función con restricciones.

Supongamos que tenemos una función que define una esfera de radio 3:
\[
    f(x, y, z) = x^2 + y^2 + z^2 = 9.
\]
Si nos ubicamos en un punto arbitrario de la esfera, y decidimos movernos permanentemente en linea recta 
en alguna dirección... eventualmente nuestros pies saldrán de la esfera, y quedaremos flotando. \\

Aquí es donde entra el concepto de \textbf{planitud local}. Si hicieramos un zoom potente al punto en el que
estamos parados, la superficie de la esfera se vería perfectamente plana. Este ``plano'' imaginario sobre
el que podemos movernos \textbf{infinitesimalmente} sin salir de la esfera es el Espacio Tangente a la esfera en ese punto.\\

Lo que queremos es encontrar los movimientos que nos permitan permanecer sobre la superficie de la esfera.
Para ello, utilizamos el gradiente $\nabla f$:
\[
    \nabla f(x, y, z) = \left( \frac{\partial f}{\partial x}, \frac{\partial f}{\partial y}, \frac{\partial f}{\partial z} \right) = (2x, 2y, 2z).
\]
El gradiente apunta en la dirección perpendicular al Espacio Tangente en el punto $(x, y, z)$.
Por lo tanto, si queremos movernos sobre ese ``suelo'' sin salirnos, el movimiento debe ser perpendicular al gradiente. 
Es decir, si $\delta c = (\delta x, \delta y, \delta z)$ es nuestro movimiento, entonces debe cumplirse que:
\[
    \nabla f(x, y, z) \cdot \delta c = 0.
\]

Esto asegura que el movimiento es válido ``instantaneamente''; si queremos movimientos largos, debemos
recalcular el plano en cada paso.

\subsection{La Matriz de Contacto}
Llevando el ejemplo del gradiente a nuestras restricciones, tenemos que cada restricción de contacto
entre dos discos $c_i$ y $c_j$ forma una superficie cuya intersección restringe las direcciones en las que podemos
perturbar la configuración sin salir de la clase de contacto $\mathcal{C}(G)$. \\

Al calcular el gradiente de cada una de estas funciones de restricción, obtenemos un conjunto de vectores
que representan las direcciones en las que la distancia entre los discos $c_i$ y $c_j$ cambia más rápidamente.
Estos vectores forman las filas de una matriz llamada la \textbf{Matriz de Contacto} $A(c)$. 
Cada fila de $A(c)$ corresponde a una restricción de contacto, y cada columna corresponde a una 
coordenada de los discos en la configuración.\\

El espacio tangente $\text{Roll}(c)$ es entonces el núcleo (kernel) de la Matriz de Contacto:
\[
    \text{Roll}(c) = \text{ker}(A(c)) = T_c \mathcal{C}(G).
\]

Esto significa que cualquier vector $\delta c$ en el espacio tangente satisface la ecuación
\begin{equation}
    A(c) \cdot \delta c = 0.
\end{equation}

Es decir, las variaciones admisibles son aquellas que no alteran las restricciones de contacto
impuestas por el grafo $G$.\\

Siguiendo esta definición, la matriz de contacto $A$ tiene la forma:
\[
    A(c) = \begin{bmatrix}
    \nabla h(c)_1 \\
    \nabla h(c)_2 \\
    ... \\
    \nabla h(c)_n
    \end{bmatrix}
\]

Donde cada $\nabla h(c)_k = ||c_j -c_i||-2$ es el gradiente de la función de restricción correspondiente al 
contacto entre dos discos $c_i$ y $c_j$.\\


\subsubsection{Deducción geométrica del gradiente de restricciones}
En nuestro estudio, en lugar de derivar analíticamente cada función de restricción para obtener su gradiente,
utilizamos las caracteristicas geométricas de los discos en contacto para deducir la estructura del gradiente.\\

Consideremos dos discos en contacto, $c_i$ y $c_j$ y definamos el vector unitario que apunta en la dirección del contacto
como $u_{ij}= \frac{c_j - c_i}{||c_j - c_i||}$. Si aplicamos perturbaciones infinitesimales $\delta c_i$ y $\delta c_j$ a los centros de los discos,
la unica manera de que el contacto se mantenga es que los desplazamientos a lo largo de la línea que une los centros se cancelen mutuamente. Es decir,
la proyección del movimiento de $c_i$ sobre $u_{ij}$ debe ser igual a la proyección del movimiento de $c_j$ sobre $u_{ij}$:
\[
    \langle u_{ij}, \delta c_j \rangle = \langle u_{ij}, \delta c_i \rangle.
\]

Ilustramos esta condición en la siguiente figura:
\begin{center}
    \begin{tikzpicture}[scale=2, >=Stealth, thick]
        % --- Definición de Coordenadas ---
        \coordinate (C1) at (0,0);      % Centro Disco 1
        \coordinate (C2) at (2,0);      % Centro Disco 2
        \coordinate (Contact) at (1,0); % Punto de contacto
    
        % --- Parámetros de Dibujo ---
        \def\R{1}          % Radio de los discos
        \def\proyLen{0.6}  % Longitud de la proyección
        
        % Vectores perturbación
        \coordinate (dC1) at (\proyLen, 0.5); 
        \coordinate (dC2) at ($(C2) + (\proyLen, -0.4)$);
    
        % --- 1. Discos (Fondo) ---
        \draw[black, fill=white, fill opacity=1.0] (C1) circle (\R);
        \draw[black, fill=white, fill opacity=1.0] (C2) circle (\R);
        
        % --- 2. Línea de Centros ---
        \draw[dashed, orange!60, thin] (-0.8,0) -- (2.8,0);
    
        % Centros
        \fill[black] (C1) circle (1.5pt) node[below left] {$c_i$};
        \fill[black] (C2) circle (1.5pt) node[below left] {$c_j$};
    
        % --- 3. Vector Unitario u_ij ---
        % Desplazado ligeramente hacia abajo para no chocar
        \draw[->, pastelgreen, line width=1pt] 
            ($(C1)+(0, -0.15)$) -- ($(Contact)+(0, -0.15)$) 
            node[midway, below=1pt] {$u_{ij}$};
    
        % --- 4. Perturbación Disco 1 (Izquierda) ---
        % Vector delta c_i
        \draw[->, cyan, line width=1pt] (C1) -- (dC1) 
            node[midway, above left=-2pt] {$\delta c_i$};
        
        % Proyección
        \draw[->, orange, line width=1pt] (C1) -- (\proyLen, 0) 
            node[midway, above=-1pt, font=\scriptsize] {$\text{p}_i$};
        
        % Línea de caída
        \draw[densely dotted, orange, thick] (dC1) -- (\proyLen, 0);
    
        % --- 5. Perturbación Disco 2 (Derecha) ---
        % Vector delta c_j
        \draw[->, cyan, line width=1pt] (C2) -- (dC2) 
            node[midway, below left=-4pt] {$\delta c_j$};
        
        % Proyección
        \draw[->, orange, line width=1pt] (C2) -- ($(C2)+(\proyLen, 0)$) 
            node[midway, above=-1pt, font=\scriptsize] {$\text{p}_j$};
        
        % Línea de caída
        \draw[densely dotted, orange, thick] (dC2) -- ($(C2)+(\proyLen, 0)$);
    
    \end{tikzpicture}
\end{center}
Donde $\text{p}_i = \langle u_{ij}, \delta c_i \rangle$ y $\text{p}_j = \langle u_{ij}, \delta c_j \rangle$ son las proyecciones
de las perturbaciones sobre el vector unitario $u_{ij}$.\\

Si reordenamos los términos, obtenemos:
\begin{equation}
    \langle u_{ij}, \delta c_j \rangle - \langle u_{ij}, \delta c_i \rangle = 0 \quad \Rightarrow \quad \langle u_{ij}, \delta c_j - \delta c_i \rangle = 0.
    \label{eq:condicion-contacto-primer-orden}
\end{equation}

Que constituye exactamente el gradiente de la restricción de contacto entre los discos $c_i$ y $c_j$. Al expandir el producto interno:
\[
    \langle u_{ij}, \delta c_j - \delta c_i \rangle = u_{ij} \cdot \delta c_j - u_{ij} \cdot \delta c_i = 0,
\]
se revela que los coeficientes que acompañan a las variaciones de la restricción son
$-u_{ij}$ para el disco $c_i$ y $+u_{ij}$ para el disco $c_j$.\\

\subsubsection{Estructura de la Matriz de Contacto}

Considerando lo anterior, si descomponemos el vector unitario $u_{ij}$ en sus componentes:
\[
    u_{ij} = \frac{c_j - c_i}{||c_j - c_i||} = \begin{bmatrix} u^x_{ij} \\[0.5em] u^y_{ij} \end{bmatrix},
\]
podemos expresar explícitamente la estructura de la fila $k$ de la matriz $A(c)$. Recordemos primero que el 
vector de configuración $c$ intercala las coordenadas $x$ e $y$ de cada disco:
\[
    c = (\dots, \underbrace{x_i, y_i}_{c_i}, \dots, \underbrace{x_j, y_j}_{c_j}, \dots).
\]
Si existe una restricción de contacto entre los discos $c_i$ y $c_j$, entonces la fila correspondiente
de la matriz de contacto $A(c)$ se define como:
\begin{equation}
    \text{row}(A(c))_k = \Big[ 
    \cdots \mathbf{0} \cdots,
    \underbrace{-u^x_{ij}, \ -u^y_{ij}}_{\text{Cols. } x_i, y_i}, 
    \cdots \mathbf{0} \cdots, 
    \underbrace{u^x_{ij}, \ u^y_{ij}}_{\text{Cols. } x_j, y_j}, 
    \cdots \mathbf{0} \cdots 
    \Big].
    \label{eq:gradiente-restriccion}    
\end{equation}

Notemos que:
\begin{itemize}
    \item Las columnas correspondientes al disco $c_i$ contienen las componentes del vector unitario con signo negativo ($-u_{ij}$).
    \item Las columnas correspondientes al disco $c_j$ contienen las componentes con signo positivo ($+u_{ij}$).
    \item Todas las demás columnas (correspondientes a discos no involucrados en este contacto específico) son cero.
\end{itemize}

\subsubsection{Ejemplo}
Construyamos la matriz de contacto para la configuración triangular. Consideremos los siguientes datos:

\begin{center}
    \begin{tabular}{ccc}
    \hline
    \textbf{Disco} & \textbf{Centro (x, y)} & \textbf{Restricción} \\
    \hline
    $c_1$ & (0, 0)          & $||c_2-c_1|| - 2 = 0$ \\
    $c_2$ & (2, 0)          & $||c_3-c_2|| - 2 = 0$ \\
    $c_3$ & (1, $\sqrt{3}$) & $||c_1-c_3|| - 2 = 0$ \\
    \hline
    \end{tabular}
\end{center}

Tenemos 3 restricciones y 6 variables, por lo que la matriz de contacto $A(c)$ será de tamaño $3 \times 6$.

\textbf{Paso 1: Cálculo de vectores unitarios} \\
Primero calculamos los vectores de dirección $u_{ij}$ para cada contacto:
\[
    u_{12} = \dfrac{c_2 - c_1}{||c_2 - c_1||} = (1, 0)
\]
\[
    u_{23} = \dfrac{c_3 - c_2}{||c_3 - c_2||} = \left(-\dfrac{1}{2}, \dfrac{\sqrt{3}}{2}\right)
\]
\[
    u_{31} = \dfrac{c_1 - c_3}{||c_1 - c_3||} = \left(-\dfrac{1}{2}, -\dfrac{\sqrt{3}}{2}\right)
\]

\textbf{Paso 2: Construcción de las filas} \\
Aplicando la estructura definida en la ecuación \eqref{eq:gradiente-restriccion}, ensamblamos cada fila asignando los componentes negativos y positivos a las columnas de los discos correspondientes:

\begin{align*}
    \text{row}(A)_1 &= \Big[ \underbrace{-1, \ 0}_{\text{Cols. } c_1}, \ \underbrace{1, \ 0}_{\text{Cols. } c_2}, \ 0, \ 0 \Big] \\[1em]
    \text{row}(A)_2 &= \Big[ 0, \ 0, \ \underbrace{\frac{1}{2}, \ -\frac{\sqrt{3}}{2}}_{\text{Cols. } c_2}, \ \underbrace{-\frac{1}{2}, \ \frac{\sqrt{3}}{2}}_{\text{Cols. } c_3} \Big] \\[1em]
    \text{row}(A)_3 &= \Big[ \underbrace{-\frac{1}{2}, \ -\frac{\sqrt{3}}{2}}_{\text{Cols. } c_1}, \ 0, \ 0, \ \underbrace{\frac{1}{2}, \ \frac{\sqrt{3}}{2}}_{\text{Cols. } c_3} \Big]
\end{align*}

\textbf{Paso 3: Matriz de Contacto completa} \\
Apilando estas filas, obtenemos la matriz final. Incluimos encabezados para identificar bien las contribuciones para cada disco:

\renewcommand{\arraystretch}{1.3}
\[
    A(c) = \left[
    \begin{array}{cc|cc|cc}
    \multicolumn{2}{c}{c_1} & \multicolumn{2}{c}{c_2} & \multicolumn{2}{c}{c_3} \\
    x_1 & y_1 & x_2 & y_2 & x_3 & y_3 \\ \hline
    -1 & 0 & 1 & 0 & 0 & 0 \\
    0 & 0 & \frac{1}{2} & -\frac{\sqrt{3}}{2} & -\frac{1}{2} & \frac{\sqrt{3}}{2} \\
    -\frac{1}{2} & -\frac{\sqrt{3}}{2} & 0 & 0 & \frac{1}{2} & \frac{\sqrt{3}}{2}
    \end{array}
    \right]
\]
\renewcommand{\arraystretch}{1}

\subsection{Condición de criticidad}
Volvamos a la condición de criticidad \eqref{eq:criticidad} que queremos satisfacer:
\[
    D \text{Per}(c)[\delta c] = 0
\]

Como ya sabemos que $\delta c$ es un vector que existe en el espacio de restricciones $\text{Roll}(c)$,
podemos reformular la condición de criticidad como:
\begin{equation}
    D \text{Per}(c)[\delta c] = 0, \quad \forall \delta c \in \text{Roll}(c).
    \label{eq:criticidad-extended}
\end{equation}

Esto significa que la primera variación del perímetro debe ser cero para cualquier perturbación
dentro del espacio tangente a la clase de contacto en el punto $c$.\\

¿Cuál es exactamente la forma de la primera derivada del perímetro? La calcularemos explicitamente a continuación.

\subsection{La primera variación del perímetro}
Recordemos que el perímetro está definido como:
\begin{equation}
    \text{Per}(c) = \sum_{(u,v) \in \mathcal{B}} ||c_v - c_u|| + 2\pi.
\end{equation}
Para calcular la primera variación del perímetro, aplicamos la definición de derivada direccional:

\[
    D \text{Per}(c)[\delta c] = \frac{d}{dt} \text{Per}(c + t \delta c).
\]
Por definición, la forma de la derivada direccional en nuestro caso es:
\begin{equation}
    D \text{Per}(c)[\delta c] = \langle \nabla \text{Per}(c), \delta c \rangle.
    \label{eq:derivada-direccional}
\end{equation}
Procederemos a calcular la derivada explícitamente.\\

Al sustituir el parametro $c + t \delta c$ en la expresión del perímetro, obtenemos:
\[
    \text{Per}(c + t \delta c) = \sum_{(u,v) \in \mathcal{B}} ||(c_v + t \delta c_v) - (c_u + t \delta c_u)|| + 2\pi.
\]
Y luego, la forma de la derivada es:
\[
    \frac{d}{dt} \text{Per}(c + t \delta c) = \frac{d}{dt} \sum_{(u,v) \in \mathcal{B}} ||(c_v + t \delta c_v) - (c_u + t \delta c_u)||.
\]
Sabemos que la derivada de la sumatoria es la suma de las derivadas, por lo que podemos centrarnos solo en:
\[
    \frac{d}{dt} ||(c_v + t \delta c_v) - (c_u + t \delta c_u)||.
\]
Para conveniencia, reordenaremos los terminos para que separar los $c$ de los $\delta c$:
\[
    \frac{d}{dt} ||(c_v - c_u) + t (\delta c_v - \delta c_u)||.
\]
Y hacemos que $x = (c_v - c_u) + t (\delta c_v - \delta c_u)$:
\[
    \frac{d}{dt} ||x(t)|| = \frac{d}{dt} \sqrt{x_1(t)^2 + x_2(t)^2}.
\]
Por regla de la cadena (f(g(x))' = f'(g(x))g'(x)):
\[
    \frac{d}{dt} ||x(t)|| = \frac{1}{2}(x_1(t)^2 + x_2(t)^2)^{-\frac{1}{2}} \cdot 2(x_1(t)x_1'(t) + x_2(t)x_2'(t)).
\]
Simplificamos la fracción y reescribimos:
\[
    \frac{d}{dt} ||x(t)|| = \frac{x_1(t)x_1'(t) + x_2(t)x_2'(t)}{||x(t)||}.
\]
Notamos que el numerador tiene la forma de un producto interno ($a_1b_1 + a_2b_2 = \langle a,b \rangle$):
\[
    \frac{d}{dt} ||x(t)|| = \frac{\langle x(t),x'(t)\rangle}{||x(t)||}.
\]
La bilinealidad del producto interno nos permite escribir:
\[
    \frac{d}{dt} ||x(t)|| = \left\langle \frac{x(t)}{||x(t)||}, x'(t) \right\rangle.
\]
Finalmente, recordamos que $x(t) = (c_v - c_u) + t (\delta c_v - \delta c_u)$ y $x'(t) = (\delta c_v - \delta c_u)$, por lo que:
\[
    \frac{d}{dt} ||(c_v + t \delta c_v) - (c_u + t \delta c_u)|| = \left\langle \frac{(c_v - c_u) + t (\delta c_v - \delta c_u)}{||(c_v - c_u) + t (\delta c_v - \delta c_u)||}, (\delta c_v - \delta c_u) \right\rangle.
\]
Evaluando en $t=0$, obtenemos la expresión final para la primera variación del
perímetro:
\[
    D \text{Per}(c)[\delta c] = \sum_{(u,v) \in \mathcal{B}} \left\langle \frac{(c_v - c_u)}{||c_v - c_u||}, (\delta c_v - \delta c_u) \right\rangle.
\]
Si consideramos el primer termino del producto interno como un vector unitario $t_{uv} = \dfrac{(c_v - c_u)}{||c_v - c_u||}$,
la expresión se puede reescribir como:
\begin{equation}
    D \text{Per}(c)[\delta c] = \sum_{(u,v) \in \mathcal{B}} \langle t_{uv}, (\delta c_v - \delta c_u) \rangle.
    \label{eq:first-variation-perimeter}
\end{equation}

Si bien la expresión se ``parece' a (\ref{eq:derivada-direccional}), no es exactamente igual. Para llegar a esa forma,
necesitamos reordenar los la sumatoria para nuestra conveniencia. Consideramos en primer lugar separar 
el producto interno en dos terminos:
\[
    D \text{Per}(c)[\delta c] = \sum_{(u,v) \in \mathcal{B}} \langle t_{uv}, \delta c_v \rangle - \sum_{(u,v) \in \mathcal{B}} \langle t_{uv}, \delta c_u \rangle.
\]
Luego, reordenamos cada sumatoria para agrupar los terminos por disco:
\[
    D \text{Per}(c)[\delta c] = \sum_{(u,k) \in \mathcal{B}} \langle t_{uk}, \delta c_k \rangle - \sum_{(k,v) \in \mathcal{B}} \langle t_{kv}, \delta c_k \rangle.
\]
Finalmente, combinamos las dos sumatorias en una sola:
\begin{equation}
    D \text{Per}(c)[\delta c] = \sum_{k \in \mathcal{V}} \langle t_{in} - t_{out}, \delta c_k \rangle.
    \label{eq:first-variation-reordered}
\end{equation}
Donde $t_{in}$ es el vector tangente que entra al disco $c_k$ y $t_{out}$ es el vector tangente que sale del disco $c_k$. $\mathcal{V}$ es el conjunto de 
discos que forman parte de la envolvente convexa. $t_{in}$ y $-t_{out}$  permiten representar la contribución neta
de cada disco al gradiente del perímetro. Al efectuar la sumatoria, la forma final es exactamente igual a la ecuación (\ref{eq:derivada-direccional}).\\

Finalmente, tenemos que la condición de criticidad \eqref{eq:criticidad-extended} se puede reescribir como\footnote{Como $\sum_{(u,k) \in \mathcal{B}} \langle t_{uv}, \delta c_v - \delta c_u \rangle = \langle \nabla \text{Per}(c), \delta c \rangle$, es correcto decir que la suma de $t_{uv}$ es, de hecho, el gradiente.}:
\begin{equation}
    \langle \nabla \text{Per}(c), \delta c \rangle = 0, \quad \forall \delta c \in \text{Roll}(c).
    \label{eq:criticidad-final}
\end{equation}
Equivalentemente, la operación corresponde a restringir el gradiente al espacio tangente, de manera que el vector
resultante sea nulo:
\[
    \text Z^\top \nabla \text{Per}(c) = 0,
\]
donde $Z$ es una base del espacio $\text{Roll}(c)$.



\subsubsection{Ejemplo}
Consideremos nuevamente la configuración triangular de antes. Recordando sus datos:
% tabla con discos, centros y vectores unitarios u
\begin{center}
    \begin{tabular}{ccc}
    \hline
    \textbf{Disco} & \textbf{Centro (x, y)} & \textbf{Vector Unitario $t_{in}$ / $t_{out}$} \\
    \hline
    $c_1$ & (0, 0)          & $u_{ij} = (1, 0)$ \\
    $c_2$ & (2, 0)          & $u_{ij} = (-\frac{1}{2}, \frac{\sqrt{3}}{2})$ \\
    $c_3$ & (1, $\sqrt{3}$) & $u_{ij} = (-\frac{1}{2}, -\frac{\sqrt{3}}{2})$ \\
    \hline
    \end{tabular}
\end{center}

Los contactos coinciden con los bordes de la envolvente convexa, por lo que cada vector t es:
\begin{itemize}
    \item Para el disco $c_1$: $t_{in} = \left(-\frac{1}{2}, -\frac{\sqrt{3}}{2}\right)$, $t_{out} = (1, 0)$
    \item Para el disco $c_2$: $t_{in} = (1, 0)$, $t_{out} = \left(-\frac{1}{2}, \frac{\sqrt{3}}{2}\right)$
    \item Para el disco $c_3$: $t_{in} = \left(-\frac{1}{2}, \frac{\sqrt{3}}{2}\right)$, $t_{out} = \left(-\frac{1}{2}, -\frac{\sqrt{3}}{2}\right)$
\end{itemize}

Luego, los gradientes netos para cada disco son:
\[
    \nabla \text{Per}(c_1) = t_{in} - t_{out} = \left(-\frac{1}{2}, -\frac{\sqrt{3}}{2}\right) - (1, 0) = \left(-\frac{3}{2}, -\frac{\sqrt{3}}{2}\right)
\]
\[
    \nabla \text{Per}(c_2) = t_{in} - t_{out} = (1, 0) - \left(-\frac{1}{2}, \frac{\sqrt{3}}{2}\right) = \left(\frac{3}{2}, -\frac{\sqrt{3}}{2}\right)
\]
\[
    \nabla \text{Per}(c_3) = t_{in} - t_{out} = \left(-\frac{1}{2}, \frac{\sqrt{3}}{2}\right) - \left(-\frac{1}{2}, -\frac{\sqrt{3}}{2}\right) = (0, \sqrt{3})
\]
Por lo tanto, el gradiente completo del perímetro es:
\[
    \nabla \text{Per}(c)^\top = \left[-\frac{3}{2}, -\frac{\sqrt{3}}{2}, \frac{3}{2}, -\frac{\sqrt{3}}{2}, 0, \sqrt{3}\right].
\]
La condición de criticidad (\ref{eq:criticidad-final}) requiere que este gradiente sea ortogonal
a cualquier variación admisible $\delta c$ en el espacio $\text{Roll}(c)$. Es decir, 
la proyección del gradiente sobre el espacio tangente debe ser cero. Si consideramos $Z$ como una
base de $\text{Roll}(c)$, entonces la condición se puede expresar como:
\[
    Z^T \nabla \text{Per}(c) = 0.
\]
\renewcommand{\arraystretch}{1.5} 
Consideramos la siguiente base:
\[
    Z = \begin{bmatrix}
    \sqrt{3} & 1 & -\sqrt{3}\\
    -1       & 0 & 2        \\
    \sqrt{3} & 1 & -\sqrt{3}\\
    1        & 0 & 0        \\
    0        & 0 & 0        \\
    0        & 0 & 1        \\
    \end{bmatrix}.
\]
Calculamos la proyección:
\[
    Z^T \nabla \text{Per}(c) = 
    \begin{bmatrix}
    \sqrt{3}  & -1 & \sqrt{3} & 1 & 0 & 0 \\
    1         & 0  & 1        & 0 & 0 & 0 \\
    -\sqrt{3} & 2  & -\sqrt{3}& 0 & 0 & 1 \\
    \end{bmatrix}
    \begin{bmatrix}
    -\frac{3}{2} \\
    -\frac{\sqrt{3}}{2} \\
    \frac{3}{2} \\
    -\frac{\sqrt{3}}{2} \\
    0 \\
    \sqrt{3} \\
    \end{bmatrix}
\]
\renewcommand{\arraystretch}{1} % Volver al estándar después si quieres

\begin{align*}
    (Z^T \nabla \text{Per})_1 &= \sqrt{3}(-\tfrac{3}{2}) + (-1)(-\tfrac{\sqrt{3}}{2}) + \sqrt{3}(\tfrac{3}{2}) + 1(-\tfrac{\sqrt{3}}{2}) + 0 + 0 = 0 \\
    (Z^T \nabla \text{Per})_2 &= 1(-\tfrac{3}{2}) + 0 + 1(\tfrac{3}{2}) + 0 + 0 + 0 = 0 \\
    (Z^T \nabla \text{Per})_3 &= -\sqrt{3}(-\tfrac{3}{2}) + 2(-\tfrac{\sqrt{3}}{2}) - \sqrt{3}(\tfrac{3}{2}) + 0 + 0 + 1(\sqrt{3}) = 0
\end{align*}

\[
    \therefore \quad Z^T \nabla \text{Per}(c) = 
    \begin{bmatrix}
    0 \\ 0 \\ 0
    \end{bmatrix}.
\]

Así, se verifica que el gradiente del perímetro es efectivamente ortogonal
al espacio tangente, confirmando que la configuración triangular es un punto crítico del funcional perímetro en su 
clase de contacto.


\section{Segunda Variación: Determinando estabilidad}
Para saber si una configuración crítica es un mínimo, máximo o punto de silla, necesitamos analizar la segunda derivada
del funcional perímetro. El objetivo de este análisis es construir una matriz Hessiana, cuyos
autovalores nos indicarán las direcciones de estabilidad e inestabilidad alrededor del punto crítico.\\

Partiendo de nuestro resultado anterior (\ref{eq:criticidad-final}), la segunda variación del perímetro
se define como:
\begin{equation}
    D^2 \text{Per}(c)[\delta c] = \frac{d}{dt} \langle \nabla \text{Per}(c), \delta c \rangle.
    \label{eq:second-variation}
\end{equation}
Similar a la primera variación, evaluamos la derivada en $t=0$ y consideramos solo
variaciones admisibles $\delta c \in \text{Roll}(c)$.\\

Al aplicar la regla del producto:
\[
    \frac{d}{dt} \langle \nabla \text{Per}(c), \delta c \rangle = \langle \frac{d}{dt} \nabla\text{Per}(c(t)), c(t) \rangle + \langle \nabla \text{Per}(c(t)), c(t)'' \rangle.
\]

Esto divide la segunda variación en dos términos. El primer término involucra los aspectos euclidianos
de la configuración, mientras que el segundo término captura las contribuciones de las restricciones de contacto que aseguran
que el movimiento se mantenga a lo largo de la trayectoria curva impuesta por $\mathcal{C}(G)$.\\


\subsection{Calculo del termino euclidiano}
Cambiamos el punto de partida de la derivada para mayor conveniencia. Utilizamos la ecuación (\ref{eq:first-variation-perimeter}):
\[
    D^2 \text{Per}(c)[\delta c] = \frac{d}{dt} \sum_{(u,v) \in \mathcal{B}} \langle t_{uv}, (\delta c_v - \delta c_u) \rangle.
\]
Aplicando la regla del producto:
\[
    D^2 \text{Per}(c)[\delta c] = \sum_{(u,v) \in \mathcal{B}} \left( \langle \frac{d}{dt} t_{uv}, (\delta c_v - \delta c_u) \rangle + \langle t_{uv}, \frac{d}{dt}(\delta c_v - \delta c_u) \rangle \right).
\]
Nos enfocamos en el primer termino. Recordando que $t_{uv} = \dfrac{(c_v - c_u)}{||c_v - c_u||}$, y definiendo $x = c_v - c_u$, calculamos su derivada:
\[
    \frac{d}{dt} t_{uv} = \frac{d}{dt} \left( \frac{x(t)}{||x(t)||} \right).
\]
Por regla del cuociente:
\[
     \frac{x'(t) \cdot ||x(t)|| - x(t) \cdot \frac{d}{dt} ||x(t)||}{||x(t)||^2}.
\]
A partir del resultado anterior, sabemos que $\frac{d}{dt} ||x(t)|| = \left\langle t_{uv}, x'(t) \right\rangle$. Sustituyendo, y simplficando:
\[
    = \frac{x'(t)}{||x(t)||} - \frac{x(t)}{||x(t)||^2} \left\langle t_{uv}, x'(t) \right\rangle.
\]
Notamos que en el segundo termino tenemos la forma $\frac{x(t)}{||x(t)||}$, que corresponde a $t_{uv}$:
\[
    = \frac{x'(t)}{||x(t)||} - \frac{t_{uv} \left\langle t_{uv}, x'(t) \right\rangle}{||x(t)||}.
\]
\[
    = \frac{1}{||x(t)||} \left( x'(t) - t_{uv} \left\langle t_{uv}, x'(t) \right\rangle \right).
\]
Al extraer el término $x'(t)$, reconocemos la forma del operador proyección ortogonal:
\[
    = \frac{1}{||x(t)||} \left( I - t^\top_{uv} t_{uv} \right) x'(t).
\]
Finalmente, recordamos que $x'(t) = \delta c_v - \delta c_u$ y $||x(t)|| = ||c_v - c_u||$:
\begin{equation}
    = \frac{1}{||c_v - c_u||} \left( I - t^\top_{uv} t_{uv} \right) (\delta c_v - \delta c_u).
    \label{eq:derivative-tangent-vector}
\end{equation}
Sustituyendo este resultado en el primer término de la segunda variación, obtenemos:
\[
    \left\langle \frac{1}{||c_v - c_u||} \left( I - t^\top_{uv} t_{uv} \right) (\delta c_v - \delta c_u), (\delta c_v - \delta c_u) \right\rangle.
\]
\[
    = \frac{1}{||c_v - c_u||} \langle (\delta c_v - \delta c_u), \left( I - t^\top_{uv} t_{uv} \right) (\delta c_v - \delta c_u) \rangle.
\]
Al resolver esta expresión, obtenemos un escalar que representa la contribución euclidiana.
No nos interesa realmente un escalar, sino una matriz que nos permita construir la Hessiana del sistema. Por lo tanto,
aislamos el término dentro del producto interno para formar el bloque:
\begin{equation}
    \nabla ^2 \text{Per}(c)_{(u,v)} = M_{uv} = \frac{1}{||c_v - c_u||} \left( I - t^\top_{uv} t_{uv} \right).
    \label{eq:euclidean-term}
\end{equation}
La matriz $\mathcal{H}_{eucl}$ se construye posicionando cada bloque $M_{uv}$ según una regla estilo laplaciano:

\begin{itemize}
    \item Las diagonales de los bloques correspondientes a los discos $c_u$ y $c_v$ reciben $\sum M_{uv}$.
    \item Los bloques fuera de la diagonal correspondientes a los pares $(c_u, c_v)$ reciben $-M_{uv}$.
    \item Los bloques correspondientes a pares de discos sin contacto reciben la matriz nula.
\end{itemize}

La matriz resultante $\mathcal{H}_{eucl}$ poseera dimensión $2n \times 2n$, donde $n$ es el número de discos en la configuración.

\subsection{Calculo del termino geométrico / de restricciones}
El termino $\langle \nabla \text{Per}(c(t)), c(t)'' \rangle$ captura las contribuciones de las restricciones de contacto.
Nuestro objetivo es expresar c''(t) en términos de las variaciones admisibles $\delta c$ y las propiedades geométricas
de la configuración.\\

\subsubsection{Multiplicadores de Lagrange}
Utilizaremos Multiplicadores de Lagrange para ayudarnos a resolver esto.

\begin{theorem}[Multiplicadores de Lagrange]
    Sea $\mathcal{X}$ una subvariedad conocida por medio de una función vectorial $F$. 
    Si una función $\varphi$ restringida a $\mathcal{X}$ tiene un punto crítico en $a 
    \in \mathcal{X}$, entonces existen valores $\lambda_1, ..., \lambda_m$ tal que 
    la derivada de $\varphi$ en $a$ es una combinación lineal de las derivadas de las 
    funciones de restricción:
    \[
    [D\varphi(a)] = \lambda_1[DF_1(a)]+...+\lambda_m[DF_m(a)]
    \]
\end{theorem}

En nuestro caso, la variedad $\mathcal{X}$ es la clase de contacto $\mathcal{C}(G)$, 
la función $\varphi$ es el funcional perímetro $\text{Per}(c)$, y las funciones de restricción
son las restricciones de contacto entre los discos.\\

Aplicando el teorema, tenemos que en un punto crítico $c \in \mathcal{C}(G)$, existen
valores $\lambda_{ij}$ asociados a cada restricción de contacto $h_{ij} = ||c_j-c_i||-2, (i,j) \in E(G)$, tales que:
\begin{equation}
    \nabla \text{Per}(c) = \sum_{(i,j) \in E(G)} \lambda_{ij} \nabla h_{ij}(c).
\end{equation}

Sustituyendo en el término geométrico de (\ref{eq:second-variation}):
\[
    \langle \nabla \text{Per}(c(t)), c(t)'' \rangle =  \left\langle \sum_{(i,j) \in E(G)} \lambda_{ij} \nabla h_{ij}(c), c(t)'' \right\rangle.
\]
\[
    = \sum_{(i,j) \in E(G)} \lambda_{ij} \langle \nabla h_{ij}(c), c(t)'' \rangle.
\]
\subsubsection{Despejando $\langle \nabla h_{ij}(c), c(t)'' \rangle$}
Derivaremos $h_{ij}$ dos veces respecto al tiempo. Siguiendo el resultado de la ecuación (\ref{eq:first-variation-reordered}), tenemos que la primera derivada es:
\[
    \frac{d}{dt} h_{ij}(c(t)) = \left\langle \frac{c_j(t) - c_i(t)}{||c_j(t) - c_i(t)||}, c_j'(t) - c_i'(t) \right\rangle = \langle \nabla h_{ij}(c), c'(t) \rangle.
\]
Por regla del producto, tenemos luego que la segunda derivada es:
\[
    \frac{d}{dt} \langle \nabla h_{ij}(c), c'(t) \rangle = \langle \frac{d}{dt} \nabla h_{ij}(c), c'(t) \rangle + \langle \nabla h_{ij}(c), c(t)'' \rangle.
\]
Evidenciamos en (\ref{eq:euclidean-term}) que $\frac{d}{dt} \nabla f(x)$ es $\nabla ^2 f(x) \cdot x'$. Por lo tanto:
\[
    \frac{d}{dt} \langle \nabla h_{ij}(c), c'(t) \rangle = \langle \nabla^2 h_{ij}(c) \cdot c'(t), c'(t) \rangle + \langle \nabla h_{ij}(c), c(t)'' \rangle.
\]
Reordenando para despejar $\langle \nabla h_{ij}(c), c(t)'' \rangle$:
\[
    \langle \nabla h_{ij}(c), c(t)'' \rangle = \frac{d}{dt} \langle \nabla h_{ij}(c), c'(t) \rangle - \langle \nabla^2 h_{ij}(c) \cdot c'(t), c'(t) \rangle.
\]
Notamos que la derivada del primer término es cero, ya que $h_{ij}(c) = 0$ para todo $t$ (la restricción se mantiene). Por lo tanto:
\[
    \langle \nabla h_{ij}(c), c(t)'' \rangle = - \langle \nabla^2 h_{ij}(c) \cdot c'(t), c'(t) \rangle.
\]
Evaluando en 0, y reescribiendo matricialmente:
\begin{equation}
    \langle \nabla h_{ij}(c), c''(0) \rangle = - \delta c^\top \nabla^2 h_{ij}(c) \delta c.
    \label{eq:second-derivative-restriction}
\end{equation}

\subsubsection{Matriz geométrica de restricciones}
Sustituyendo (\ref{eq:second-derivative-restriction}) en el segundo término de la segunda variación, obtenemos:
\[
    \langle \nabla \text{Per}(c(t)), c(t)'' \rangle = - \sum_{(i,j) \in E(G)} \lambda_{ij} \delta c^\top \nabla^2 h_{ij}(c) \delta c.
\]
Calculamos $\nabla^2 h_{ij}(c)$. Según lo que obtuvimos en la ecuación (\ref{eq:euclidean-term}), tenemos que:
\[
    \nabla^2 h_{ij}(c) = \frac{1}{||c_j - c_i||} \left( I - u^\top_{ij} u_{ij} \right).
\]
La distancia entre los discos en contacto es constante e igual a 2, por lo que:
\[
    \nabla^2 h_{ij}(c) = \frac{1}{2} \left( I - u^\top_{ij} u_{ij} \right).
\]
Sustituyendo en la expresión de la segunda variación:
\[
    \langle \nabla \text{Per}(c(t)), c(t)'' \rangle = - \sum_{(i,j) \in E(G)} \lambda_{ij} \delta c^\top \left( \frac{1}{2} \left( I - u^\top_{ij} u_{ij} \right) \right) \delta c.
\]
Aislamos el término cuadrático en $\delta c$ para definir la matriz geométrica de restricciones:
\begin{equation}
    \mathcal{H}_{\mathrm{geom}} = - \sum_{(i,j) \in E(G)}  \frac{\lambda_{ij}}{2} \left( I - u^\top_{ij} u_{ij} \right).
    \label{eq:geometric-term}
\end{equation}

El Hessiano geométrico se construye a partir de estos bloques, siguiendo un esquema similar al del Hessiano euclidiano:
\begin{itemize}
    \item Las diagonales de los bloques correspondientes a los discos $c_i$ y $c_j$ reciben $\sum \frac{\lambda_{ij}}{2} \left( I - u^\top_{ij} u_{ij} \right)$.
    \item Los bloques fuera de la diagonal correspondientes a los pares $(c_i, c_j)$ reciben $-\frac{\lambda_{ij}}{2} \left( I - u^\top_{ij} u_{ij} \right)$.
    \item Los bloques correspondientes a pares de discos sin contacto reciben la matriz nula.
\end{itemize}


\subsubsection{Hessiano Total y Hessiano Intrínseco}
Finalmente, combinamos los dos términos para obtener el Hessiano total del sistema:
\begin{equation}
    \mathcal{H}_{\mathrm{total}} = \mathcal{H}_{eucl} + \mathcal{H}_{\mathrm{geom}}.
    \label{eq:total-hessian}
\end{equation}

Restringimos esta matriz al espacio tangente $\text{Roll}(c)$ para obtener el Hessiano intrínseco:
\begin{equation}
    \mathcal{H}_{\mathrm{intr}} = Z^\top \mathcal{H}_{\mathrm{total}} Z
    \label{eq:intrinsic-hessian}
\end{equation}
Donde $Z$ es una base de $\text{Roll}(c)$.\\

La ecuación \eqref{eq:intrinsic-hessian} que restringe el Hessiano total a $\text{Roll}(c)$ surge 
al considerar los $\delta c$ admisibles dentro del espacio tangente que fueron obtenidos al despejar cada
bloque anteriormente.
Si definimos un vector $y \in \mathbb{R}^k$ tal que cualquier variación admisible se exprese como 
$\delta c = Z y$, entonces el término cuadrático se transforma según:
\[
    \delta c^\top \mathcal{H}_{\mathrm{total}} \delta c = (Z y)^\top \mathcal{H}_{\mathrm{total}} (Z y) = y^\top \underbrace{(Z^\top \mathcal{H}_{\mathrm{total}} Z)}_{\mathcal{H}_{\mathrm{intr}}} y.
\]
Dado que $\mathcal{H}_{\mathrm{total}}$ es una matriz de $2n \times 2n$ y $Z$ es de $2n \times k$ 
(donde $k$ es la dimensión de \text{Roll}(c)), el producto resultante $\mathcal{H}_{\mathrm{intr}}$ constituye una 
matriz cuadrada de $k \times k$. \\

El análisis de los autovalores de $\mathcal{H}_{\mathrm{intr}}$ nos permitirá determinar la estabilidad.
Los movimientos con autovalor 0 corresponden a direcciones dentro del espacio tangente que no modifican
el perímetro. Siempre encontraremos como mínimo 3 autovalores 0, correspondientes a las
traslaciones y rotación. Cualquier autovalor adicional 0 indica una dirección plana. \\

Si todos los autovalores son positivos, la configuración es un mínimo local (estable).\\

Si todos son negativos, es un máximo local (inestable). Si hay más de 3 autovalores 0, hablamos de una dirección plana.

\subsubsection{Ejemplo}
Consideraremos una configuración cuadrada de 4 discos en contacto. Los centros de los discos son:
\begin{center}
    \begin{tabular}{cc}
    \hline
    \textbf{Disco} & \textbf{Centro (x, y)} \\
    \hline
    $c_1$ & (0, 0) \\
    $c_2$ & (2, 0) \\
    $c_3$ & (2, 2) \\
    $c_4$ & (0, 2) \\
    \hline
    \end{tabular}
\end{center}

Al igual que con el triangulo, las aristas de le envolvente coinciden con los contactos:
\begin{center}
    \begin{tabular}{ccc}
    \hline
    \textbf{Vector Unitario $u_{ij}$} & \textbf{Vector Unitario $t_{uv}$} \\
    \hline
    $u_{12} = (1, 0)$  & $t_{12} = (1, 0)$ \\
    $u_{23} = (0, 1)$  & $t_{23} = (0, 1)$\\
    $u_{34} = (-1, 0)$ & $t_{34} = (-1, 0)$\\
    $u_{41} = (0, -1)$ & $t_{41} = (0, -1)$\\
    \hline
    \end{tabular}
\end{center}
La matriz de contacto de esta configuración es:
\[
    A = \begin{bmatrix}
    -1 & 0 & 1 & 0 & 0 & 0 & 0 & 0\\
    0 & 0 & 0 & -1 & 0 & 1 & 0 & 0 \\
    0 & 0 & 0 & 0 & 1 & 0 & -1 & 0 \\
    0 & -1 & 0 & 0 & 0 & 0 & 0 & 1
    \end{bmatrix}
\]

Y su gradiente:
\[
    \nabla \text{Per}(c)^\top = \begin{bmatrix}
    -1 & -1 & 1 & -1 & 1 & 1 & -1 & 1
    \end{bmatrix}
\]

Calculamos los bloques del Hessiano euclidiano utilizando la ecuación (\ref{eq:euclidean-term}).
Como $||c_j - c_i|| = 2$ para todos los contactos, tenemos:
\[
    M_{12} = \frac{1}{2} \left( 
        \begin{bmatrix} 1 & 0 \\ 0 & 1 \end{bmatrix} 
        - \begin{bmatrix} 1 \\ 0 \end{bmatrix} 
        \begin{bmatrix} 1 & 0 \end{bmatrix} \right) = \frac{1}{2} \begin{bmatrix} 0 & 0 \\ 0 & 1 \end{bmatrix}
\]
\[
    M_{23} = \frac{1}{2} \left( 
        \begin{bmatrix} 1 & 0 \\ 0 & 1 \end{bmatrix} 
        - \begin{bmatrix} 0 \\ 1 \end{bmatrix} 
        \begin{bmatrix} 0 & 1 \end{bmatrix} \right) = \frac{1}{2} \begin{bmatrix} 1 & 0 \\ 0 & 0 \end{bmatrix}
\]
\[
    M_{34} = \frac{1}{2} \left( 
        \begin{bmatrix} 1 & 0 \\ 0 & 1 \end{bmatrix} 
        - \begin{bmatrix} -1 \\ 0 \end{bmatrix} 
        \begin{bmatrix} -1 & 0 \end{bmatrix} \right) = \frac{1}{2} \begin{bmatrix} 0 & 0 \\ 0 & 1 \end{bmatrix}
\]
\[
    M_{41} = \frac{1}{2} \left( 
        \begin{bmatrix} 1 & 0 \\ 0 & 1 \end{bmatrix} 
        - \begin{bmatrix} 0 \\ -1 \end{bmatrix} 
        \begin{bmatrix} 0 & -1 \end{bmatrix} \right) = \frac{1}{2} \begin{bmatrix} 1 & 0 \\ 0 & 0 \end{bmatrix}
\]
Siguiendo la estructura de ensamblaje que mencionamos antes, podemos construir la matriz
Euclidiana de la siguiente manera:

\renewcommand{\arraystretch}{1.5}
\[
    \mathcal{H}_{eucl} = 
    \left[
    \begin{array}{c|c|c|c}
        \underbrace{M_{12} + M_{41}}_{\text{Disco } 1} & -M_{12} & 0 & -M_{41} \\ \hline
        -M_{12} & \underbrace{M_{12} + M_{23}}_{\text{Disco } 2} & -M_{23} & 0 \\ \hline
        0 & -M_{23} & \underbrace{M_{23} + M_{34}}_{\text{Disco } 3} & -M_{34} \\ \hline
        -M_{41} & 0 & -M_{34} & \underbrace{M_{34} + M_{41}}_{\text{Disco } 4}
    \end{array}
    \right]
\]
\renewcommand{\arraystretch}{1}
Sustituyendo cada bloque, obtenemos el Hessiano euclidiano:

\[
    \mathcal{H}_{eucl} = \begin{bmatrix}
    \frac{1}{2} & 0 & 0 & 0 & 0 & 0 & -\frac{1}{2} & 0 \\[0.8em]
    0 & \frac{1}{2} & 0 & -\frac{1}{2} & 0 & 0 & 0 & 0 \\[0.8em]
    0 & 0 & \frac{1}{2} & 0 & -\frac{1}{2} & 0 & 0 & 0 \\[0.8em]
    0 & -\frac{1}{2} & 0 & \frac{1}{2} & 0 & 0 & 0 & 0 \\[0.8em]
    0 & 0 & -\frac{1}{2} & 0 & \frac{1}{2} & 0 & 0 & 0 \\[0.8em]
    0 & 0 & 0 & 0 & 0 & \frac{1}{2} & 0 & -\frac{1}{2} \\[0.8em]
    -\frac{1}{2} & 0 & 0 & 0 & 0 & 0 & \frac{1}{2} & 0 \\[0.8em]
    0 & 0 & 0 & 0 & 0 & -\frac{1}{2} & 0 & \frac{1}{2}
    \end{bmatrix}
\]

Para el Hessiano geométrico, calculamos los multiplicadores de Lagrange por medio del Teorema 1.
Consideramos a la matriz de contacto A como la matriz que agrupa las derivadas de las restrcciones. Luego:
\[
    A(c)^\top \lambda = \nabla \text{Per}(c)
\]

Al resolver el sistema, obtenemos el vector columna $\lambda$:
% matriz A traspuesta * vector lambda = gradiente perímetro
\[
    \begin{bmatrix}
    -1 & 0 & 0 & 0 \\
    0 & 0 & 0 & -1 \\
    1 & 0 & 0 & 0 \\
    0 & -1 & 0 & 0 \\
    0 & 0 & 1 & 0 \\
    0 & 1 & 0 & 0 \\
    0 & 0 & -1 & 0 \\
    0 & 0 & 0 & 1    
    \end{bmatrix} \cdot
    \begin{bmatrix}
        \lambda_{12} \\
        \lambda_{23} \\
        \lambda_{34} \\
        \lambda_{41}
    \end{bmatrix} =
    \begin{bmatrix}
    -1 \\ -1 \\ 1 \\ -1 \\ 1 \\ 1 \\ -1 \\ 1
    \end{bmatrix}
\]
Para cada $\lambda$ tenemos dos ecuaciones:
\[
    \begin{cases}
        -\lambda_{12} = -1 \\
        \lambda_{12} = 1
    \end{cases}
    \quad
    \begin{cases}
        -\lambda_{23} = -1 \\
        \lambda_{23} = 1
    \end{cases}
    \quad
    \begin{cases}
        \lambda_{34} = 1 \\
        -\lambda_{34} = -1 
    \end{cases}
    \quad
    \begin{cases}
        -\lambda_{41} = -1 \\
        \lambda_{41} = 1
    \end{cases}
\]
Todas ellas son consistentes, por lo que obtenemos:
\[
    \lambda_{12} = \lambda_{23} = \lambda_{34} = \lambda_{41} = 1.
\]
Nuestro bloque del Hessiano Geometrico toma la forma:
\[
    K_{ij} = - \frac{1}{2} \left( I - u^\top_{ij} u_{ij} \right).
\]
Como en este caso $u_{ij}$ y $t_{uv}$ coinciden, los bloques son iguales a los del Hessiano Euclidiano pero con signo negativo:
\[
    K_{12} = -\frac{1}{2} \begin{bmatrix} 0 & 0 \\ 0 & 1 \end{bmatrix}, \quad
    K_{23} = -\frac{1}{2} \begin{bmatrix} 1 & 0 \\ 0 & 0 \end{bmatrix}, \quad
    K_{34} = -\frac{1}{2} \begin{bmatrix} 0 & 0 \\ 0 & 1 \end{bmatrix}, \quad
    K_{41} = -\frac{1}{2} \begin{bmatrix} 1 & 0 \\ 0 & 0 \end{bmatrix}.
\]
Apilando los bloques, obtenemos el Hessiano Geométrico completo:
\[
    \mathcal{H}_{\mathrm{geom}} = \begin{bmatrix}
    -\frac{1}{2} & 0 & 0 & 0 & 0 & 0 & \frac{1}{2} & 0 \\[0.8em]
    0 & -\frac{1}{2} & 0 & \frac{1}{2} & 0 & 0 & 0 & 0 \\[0.8em]
    0 & 0 & -\frac{1}{2} & 0 & \frac{1}{2} & 0 & 0 & 0 \\[0.8em]
    0 & \frac{1}{2} & 0 & -\frac{1}{2} & 0 & 0 & 0 & 0 \\[0.8em]
    0 & 0 & \frac{1}{2} & 0 & -\frac{1}{2} & 0 & 0 & 0 \\[0.8em]
    0 & 0 & 0 & 0 & 0 & -\frac{1}{2} & 0 & \frac{1}{2} \\[0.8em]
    \frac{1}{2} & 0 & 0 & 0 & 0 & 0 & -\frac{1}{2} & 0 \\[0.8em]
    0 & 0 & 0 & 0 & 0 & \frac{1}{2} & 0 & -\frac{1}{2}
    \end{bmatrix}
\]

A continuación calculamos el Hessiano Total sumando ambos términos. Como los bloques son iguales pero con signo contrario, el Hessiano Total es la matriz nula de $8 \times 8$:
\[
    \mathcal{H}_{\mathrm{total}} = \mathcal{H}_{eucl} + \mathcal{H}_{\mathrm{geom}} =
        \begin{bmatrix}
        \; 0 & 0 & 0 & 0 & 0 & 0 & 0 & 0 \; \\[0.4em]
        \; 0 & 0 & 0 & 0 & 0 & 0 & 0 & 0 \; \\[0.4em]
        \; 0 & 0 & 0 & 0 & 0 & 0 & 0 & 0 \; \\[0.4em]
        \; 0 & 0 & 0 & 0 & 0 & 0 & 0 & 0 \; \\[0.4em]
        \; 0 & 0 & 0 & 0 & 0 & 0 & 0 & 0 \; \\[0.4em]
        \; 0 & 0 & 0 & 0 & 0 & 0 & 0 & 0 \; \\[0.4em]
        \; 0 & 0 & 0 & 0 & 0 & 0 & 0 & 0 \; \\[0.4em]
        \; 0 & 0 & 0 & 0 & 0 & 0 & 0 & 0 \; \\[0.4em]
        \end{bmatrix}
\]
Para restringir a $\text{Roll}(c)$ consideramos una base $Z$:
\[
    Z = 
    \begin{bmatrix}
    \; 1 & 0 & 0 & 0 \; \\[0.4em]
    \; 0 & 0 & 0 & 1 \; \\[0.4em]
    \; 1 & 0 & 0 & 0 \; \\[0.4em]
    \; 0 & 1 & 0 & 0 \; \\[0.4em]
    \; 0 & 0 & 1 & 0 \; \\[0.4em]
    \; 0 & 1 & 0 & 0 \; \\[0.4em]
    \; 0 & 0 & 1 & 0 \; \\[0.4em]
    \; 0 & 0 & 0 & 1 \; \\[0.4em]
    \end{bmatrix}
\]

Al realizar la operación:


obtenemos nuevamente la matriz nula pero con la dimensión reducida ($4 \times 4$):
\[
    \mathcal{H}_{\mathrm{intr}} = Z^\top_{4 \times 8} \cdot \mathcal{H}_{\mathrm{total}_{8 \times 8}} \cdot Z_{8 \times 4} = 
    \begin{bmatrix}
    \; 0 & 0 & 0 & 0 \; \\[0.4em]
    \; 0 & 0 & 0 & 0 \; \\[0.4em]
    \; 0 & 0 & 0 & 0 \; \\[0.4em]
    \; 0 & 0 & 0 & 0 \; \\[0.4em]
    \end{bmatrix}
\]
El Hessiano intrínseco es la matriz nula de dimensión $4 \times 4$. Esto implica que tiene 4 autovalores 0,
lo que indica que la configuración cuadrada es un punto crítico con múltiples direcciones planas.

\[
    \boxed{
    \text{Autovalores de } \mathcal{H}_{\mathrm{intr}}: \quad 0, 0, 0, 0
    }   
\]

3 de estos autovalores corresponden a las simetrías de traslación y rotación, mientras que el cuarto
indica una dirección plana adicional, que corresponde a una familia de rombos con el mismo perímetro.

\section{Rombo}

Consideramos una configuración de 4 discos que forman un rombo.
\[
    c = {(0,0), (2,0), (3,\sqrt{3}), (1,\sqrt{3})}
\]
\[
    E = {(1,2), (2,3), (3,4), (4,1), (4,2)}
\]
\[
    \mathcal{B} = {(1,2), (2,3), (3,4), (4,1)}
\]

% Dibujo de la configuración rombo
\begin{center}
    \begin{tikzpicture}[scale=1]
        % Discos
        \draw[] (0,0) circle (1);
        \draw[] (2,0) circle (1);
        \draw[] (3,{sqrt(3)}) circle (1);
        \draw[] (1,{sqrt(3)}) circle (1);
        
        % Centros
        \filldraw[black] (0,0) circle (2pt) node[anchor=north east] {$c_1$};
        \filldraw[black] (2,0) circle (2pt) node[anchor=north west] {$c_2$};
        \filldraw[black] (3,{sqrt(3)}) circle (2pt) node[anchor=south] {$c_3$};
        \filldraw[black] (1,{sqrt(3)}) circle (2pt) node[anchor=north] {$c_4$};
        
        % Contactos
        \draw[thick] (0,0) -- (2,0);
        \draw[thick] (2,0) -- (3,{sqrt(3)});
        \draw[thick] (3,{sqrt(3)}) -- (1,{sqrt(3)});
        \draw[thick] (1,{sqrt(3)}) -- (0,0);
        \draw[thick, dashed] (1,{sqrt(3)}) -- (2,0); % Contacto adicional
    \end{tikzpicture}
\end{center}

Calculamos explicitamente los vectores unitarios de los contactos:

\[
    u_{12} = \frac{c_2 - c_1}{||c_2 - c_1||} = (1, 0), \quad
    u_{23} = \frac{c_3 - c_2}{||c_3 - c_2||} = \left(\frac{1}{2}, \frac{\sqrt{3}}{2}\right), \quad
    u_{34} = \frac{c_4 - c_3}{||c_4 - c_3||} = (-1, 0),
\]
\[
    u_{14} = \frac{c_4 - c_1}{||c_4 - c_1||} = \left(\frac{1}{2}, \frac{\sqrt{3}}{2}\right), \quad
    u_{24} = \frac{c_4 - c_2}{||c_4 - c_2||} = \left(-\frac{1}{2}, \frac{\sqrt{3}}{2}\right).
\]

Y de las aristas:
\[
    t_{12} = (1, 0), \quad
    t_{23} = \left(\frac{1}{2}, \frac{\sqrt{3}}{2}\right), \quad
    t_{34} = (-1, 0), \quad
    t_{41} = \left(-\frac{1}{2}, -\frac{\sqrt{3}}{2}\right).
\]

Entonces tenemos:

\begin{center}
    \begin{tabular}{ccc}
    \hline
    \textbf{Vector Unitario $u_{ij}$} & \textbf{Vector Unitario $t_{uv}$} \\
    \hline
    $u_{12} = (1, 0)$  & $t_{12} = (1, 0)$ \\
    $u_{23} = \left(\frac{1}{2}, \frac{\sqrt{3}}{2}\right)$  & $t_{23} = \left(\frac{1}{2}, \frac{\sqrt{3}}{2}\right)$\\
    $u_{34} = (-1, 0)$ & $t_{34} = (-1, 0)$\\
    $u_{14} = \left(\frac{1}{2}, \frac{\sqrt{3}}{2}\right)$ & $t_{41} = \left(-\frac{1}{2}, -\frac{\sqrt{3}}{2}\right)$\\
    $u_{24} = \left(-\frac{1}{2}, \frac{\sqrt{3}}{2}\right)$ & $\notin \mathcal{B}$\\
    \hline
    \end{tabular}
\end{center}

Denotando cada restricción de contacto como $h_{ij}$, la matriz de contacto $A(c)$ se construye como:
\[
    A(c) = \begin{array}{c|cccccccc}
        & c_{1x} & c_{1y} & c_{2x} & c_{2y} & c_{3x} & c_{3y} & c_{4x} & c_{4y} \\
        \hline
        h_{12} & -1 & 0 & 1 & 0 & 0 & 0 & 0 & 0 \\[0.8em]
        h_{23} & 0 & 0 & -\frac{1}{2} & -\frac{\sqrt{3}}{2} & \frac{1}{2} & \frac{\sqrt{3}}{2} & 0 & 0 \\[0.8em]
        h_{34} & 0 & 0 & 0 & 0 & 1 & 0 & -1 & 0 \\[0.8em]
        h_{41} & -\frac{1}{2} & -\frac{\sqrt{3}}{2} & 0 & 0 & 0 & 0 & \frac{1}{2} & \frac{\sqrt{3}}{2} \\[0.8em]
        h_{24} & 0 & 0 & -\left(-\frac{1}{2}\right) & -\frac{\sqrt{3}}{2} & 0 & 0 & -\frac{1}{2} & \frac{\sqrt{3}}{2} \\[0.8em]
    \end{array}
\]

Calculamos el gradiente con los vectores $t$:
\[
    \nabla_{c_1} \text{Per} = t_{in} - t_{out} = t_{41} - t_{12} = \left(-\frac{1}{2}, -\frac{\sqrt{3}}{2}\right) - (1, 0) = \left(-\frac{3}{2}, -\frac{\sqrt{3}}{2}\right), \quad
\]
\[
    \nabla_{c_2} \text{Per} = t_{in} - t_{out} = t_{12} - t_{23} = (1, 0) - \left(\frac{1}{2}, \frac{\sqrt{3}}{2}\right) = \left(\frac{1}{2}, -\frac{\sqrt{3}}{2}\right),
\]

\[
    \nabla_{c_3} \text{Per} = t_{in} - t_{out} = t_{23} - t_{34} = \left(\frac{1}{2}, \frac{\sqrt{3}}{2}\right) - (-1, 0) = \left(\frac{3}{2}, \frac{\sqrt{3}}{2}\right), \quad
\]
\[
    \nabla_{c_4} \text{Per} = t_{in} - t_{out} = t_{34} - t_{41} = (-1, 0) - \left(-\frac{1}{2}, -\frac{\sqrt{3}}{2}\right) = \left(-\frac{1}{2}, \frac{\sqrt{3}}{2}\right).
\]

Por lo tanto, $\nabla \text{Per}(c)^\top$ es:
\[
    \nabla \text{Per}(c)^\top = \begin{bmatrix}
    -\frac{3}{2} & -\frac{\sqrt{3}}{2} & \frac{1}{2} & -\frac{\sqrt{3}}{2} & \frac{3}{2} & \frac{\sqrt{3}}{2} & -\frac{1}{2} & \frac{\sqrt{3}}{2}
    \end{bmatrix}
\]

Usamos una base de Roll(c):
\[
    Z = 
    \begin{bmatrix}
    \; \frac{\sqrt{3}}{2}  & 1 & -\frac{\sqrt{3}}{2}  \; \\[0.4em]
    \; -\frac{1}{2}        & 0 & \frac{3}{2}          \; \\[0.4em]
    \; \frac{\sqrt{3}}{2}  & 1 & -\frac{\sqrt{3}}{2}   \; \\[0.4em]
    \; \frac{1}{2}         & 0 & \frac{1}{2}          \; \\[0.4em]
    \; 0         & 1 & 0                    \; \\[0.4em]
    \; 1         & 0 & 0                    \; \\[0.4em]
    \; 0         & 1 & 0                    \; \\[0.4em]
    \; 0         & 0 & 1                    \; \\[0.4em]
    \end{bmatrix}
\]

Luego, al hacer $Z^\top \nabla \text{Per}(c)$, obtenemos:
\[
    \begin{bmatrix}
        \; \frac{\sqrt{3}}{2} & -\frac{1}{2} & \frac{\sqrt{3}}{2} & \frac{1}{2} & 0 & 1 & 0 & 0 \; \\[0.4em]
        \; 1         & 0 & 1 & 0 & 1 & 0 & 1 & 0 \; \\[0.4em]
        \; -\frac{\sqrt{3}}{2} & \frac{3}{2} & -\frac{\sqrt{3}}{2} & \frac{1}{2} & 0 & 0 & 0 & 1 \; \\[0.4em]
    \end{bmatrix} \cdot
    \begin{bmatrix}
        -\frac{3}{2} \\[0.4em]
        -\frac{\sqrt{3}}{2} \\[0.4em]
        \frac{1}{2} \\[0.4em]
        -\frac{\sqrt{3}}{2} \\[0.4em]
        \frac{3}{2} \\[0.4em]
        \frac{\sqrt{3}}{2} \\[0.4em]
        -\frac{1}{2} \\[0.4em]
        \frac{\sqrt{3}}{2}
    \end{bmatrix} = 
    \begin{bmatrix}
        0 \\[0.4em]
        0 \\[0.4em]
        0
    \end{bmatrix}
\] 

Se verifica entonces que el rombo es crítico.

Calculamos cada bloque del Hessiano Euclidiano usando la ecuación:
\[
    M_{ij} = \frac{1}{||c_j - c_i||} \left( I - t^\top_{ij} t_{ij} \right).
\]
Así:
\[
    M_{12} = \frac{1}{2} \left( 
        \begin{bmatrix} 1 & 0 \\ 0 & 1 \end{bmatrix} 
        - \begin{bmatrix} 1 \\ 0 \end{bmatrix} 
        \begin{bmatrix} 1 & 0 \end{bmatrix} \right) = \frac{1}{2} \begin{bmatrix} 0 & 0 \\ 0 & 1 \end{bmatrix}
        = \begin{bmatrix} 0 & 0 \\ 0 & \frac{1}{2} \end{bmatrix}
\]
\[
    M_{23} = \frac{1}{2} \left( 
        \begin{bmatrix} 1 & 0 \\ 0 & 1 \end{bmatrix} 
        - \begin{bmatrix} \frac{1}{2} \\ \frac{\sqrt{3}}{2} \end{bmatrix} 
        \begin{bmatrix} \frac{1}{2} & \frac{\sqrt{3}}{2} \end{bmatrix} \right) 
        = \frac{1}{2} \begin{bmatrix} 1 - \frac{1}{4} & -\frac{\sqrt{3}}{4} \\ -\frac{\sqrt{3}}{4} & 1 - \frac{3}{4} \end{bmatrix}
        = \frac{1}{2} \begin{bmatrix} \frac{3}{4} & -\frac{\sqrt{3}}{4} \\ -\frac{\sqrt{3}}{4} & \frac{1}{4} \end{bmatrix}
        = \begin{bmatrix} \frac{3}{8} & -\frac{\sqrt{3}}{8} \\ -\frac{\sqrt{3}}{8} & \frac{1}{8} \end{bmatrix}
\]
\[
    M_{34} = \frac{1}{2} \left( 
        \begin{bmatrix} 1 & 0 \\ 0 & 1 \end{bmatrix} 
        - \begin{bmatrix} -1 \\ 0 \end{bmatrix} 
        \begin{bmatrix} -1 & 0 \end{bmatrix} \right) = \frac{1}{2} \begin{bmatrix} 0 & 0 \\ 0 & 1 \end{bmatrix}
        = \begin{bmatrix} 0 & 0 \\ 0 & \frac{1}{2} \end{bmatrix}
\]
\[
    M_{41} = \frac{1}{2} \left( 
        \begin{bmatrix} 1 & 0 \\ 0 & 1 \end{bmatrix} 
        - \begin{bmatrix} -\frac{1}{2} \\ -\frac{\sqrt{3}}{2} \end{bmatrix} 
        \begin{bmatrix} -\frac{1}{2} & -\frac{\sqrt{3}}{2} \end{bmatrix} \right) 
        = \frac{1}{2} \begin{bmatrix} 1 - \frac{1}{4} & -\frac{\sqrt{3}}{4} \\ -\frac{\sqrt{3}}{4} & 1 - \frac{3}{4} \end{bmatrix}
        = \frac{1}{2} \begin{bmatrix} \frac{3}{4} & -\frac{\sqrt{3}}{4} \\ -\frac{\sqrt{3}}{4} & \frac{1}{4} \end{bmatrix}
        = \begin{bmatrix} \frac{3}{8} & -\frac{\sqrt{3}}{8} \\ -\frac{\sqrt{3}}{8} & \frac{1}{8} \end{bmatrix}
\]

Por lo tanto:

\[
    M_{12} = \begin{bmatrix} 0 & 0 \\ 0 & \frac{1}{2}  \end{bmatrix}, \quad
    M_{23} = \begin{bmatrix} \frac{3}{8} & -\frac{\sqrt{3}}{8} \\ -\frac{\sqrt{3}}{8} & \frac{1}{8} \end{bmatrix}, \quad
    M_{34} = \begin{bmatrix} 0 & 0 \\ 0 & \frac{1}{2}  \end{bmatrix}, \quad
    M_{41} = \begin{bmatrix} \frac{3}{8} & -\frac{\sqrt{3}}{8} \\ -\frac{\sqrt{3}}{8} & \frac{1}{8} \end{bmatrix}.
\]

Ensamblando, obtenemos la forma:
\renewcommand{\arraystretch}{1.5}
\[
    \mathcal{H}_{eucl} = 
    \left[
    \begin{array}{c|c|c|c}
        \underbrace{M_{12} + M_{41}}_{\text{Disco } 1} & -M_{12} & 0 & -M_{41} \\ \hline
        -M_{12} & \underbrace{M_{12} + M_{23}}_{\text{Disco } 2} & -M_{23} & 0 \\ \hline
        0 & -M_{23} & \underbrace{M_{23} + M_{34}}_{\text{Disco } 3} & -M_{34} \\ \hline
        -M_{41} & 0 & -M_{34} & \underbrace{M_{34} + M_{41}}_{\text{Disco } 4}
    \end{array}
    \right]
\]
\renewcommand{\arraystretch}{1}

Sustituyendo cada bloque, obtenemos el Hessiano euclidiano:
\[
    \mathcal{H}_{eucl} = \begin{bmatrix}
    \frac{3}{8} & -\frac{\sqrt{3}}{8} & 0 & 0 & 0 & 0 & -\frac{3}{8} & \frac{\sqrt{3}}{8} \\[0.8em]
    -\frac{\sqrt{3}}{8} & \frac{5}{8} & 0 & -\frac{1}{2} & 0 & 0 & \frac{\sqrt{3}}{8} & -\frac{1}{8} \\[0.8em]
    0 & 0 & \frac{3}{8} & -\frac{\sqrt{3}}{8} & -\frac{3}{8} & \frac{\sqrt{3}}{8} & 0 & 0 \\[0.8em]
    0 & -\frac{1}{2} & -\frac{\sqrt{3}}{8} & \frac{5}{8} & \frac{\sqrt{3}}{8} & -\frac{1}{8} & 0 & 0 \\[0.8em]
    0 & 0 & -\frac{3}{8} & \frac{\sqrt{3}}{8} & \frac{3}{8} & -\frac{\sqrt{3}}{8} & 0 & 0 \\[0.8em]
    0 & 0 & \frac{\sqrt{3}}{8} & -\frac{1}{8} & -\frac{\sqrt{3}}{8} & \frac{5}{8} & 0 & -\frac{1}{2} \\[0.8em]
    -\frac{3}{8} & \frac{\sqrt{3}}{8} & 0 & 0 & 0 & 0 & \frac{3}{8} & -\frac{\sqrt{3}}{8} \\[0.8em]
    \frac{\sqrt{3}}{8} & -\frac{1}{8} & 0 & 0 & 0 & -\frac{1}{2} & -\frac{\sqrt{3}}{8} & \frac{5}{8}
    \end{bmatrix}
\]

Los bloques del Hessiano Geométrico por otra parte:
\[
    K_{ij} = - \frac{\lambda_{ij}}{2} \left( I - u^\top_{ij} u_{ij} \right).
\]
Calculamos los multiplicadores de Lagrange como antes:
\[
    A(c)^\top \lambda = \nabla \text{Per}(c)
\]

\[
\begin{bmatrix}
    -1 & 0 & 0 & -\frac{1}{2} & 0 \\[0.4em]
    0 & 0 & 0 & -\frac{\sqrt{3}}{2} & 0 \\[0.4em]
    1 & -\frac{1}{2} & 0 & 0 & \frac{1}{2} \\[0.4em]
    0 & -\frac{\sqrt{3}}{2} & 0 & 0 & -\frac{\sqrt{3}}{2} \\[0.4em]
    0 & \frac{1}{2} & 1 & 0 & 0 \\[0.4em]
    0 & \frac{\sqrt{3}}{2} & 0 & 0 & 0 \\[0.4em]
    0 & 0 & -1 & \frac{1}{2} & -\frac{1}{2} \\[0.4em]
    0 & 0 & 0 & \frac{\sqrt{3}}{2} & \frac{\sqrt{3}}{2}
\end{bmatrix} \cdot
\begin{bmatrix}
    \lambda_{12} \\
    \lambda_{23} \\
    \lambda_{34} \\
    \lambda_{41} \\
    \lambda_{24}
\end{bmatrix} =
\begin{bmatrix}
    -\frac{3}{2} \\[0.4em]
    -\frac{\sqrt{3}}{2} \\[0.4em]
    \frac{1}{2} \\[0.4em]
    -\frac{\sqrt{3}}{2} \\[0.4em]
    \frac{3}{2} \\[0.4em]
    \frac{\sqrt{3}}{2} \\[0.4em]
    -\frac{1}{2} \\[0.4em]
    \frac{\sqrt{3}}{2}
\end{bmatrix}
\]

Resolviendo el sistema, el vector solución es:
\[
    \lambda = \begin{bmatrix}
    1 \\
    1 \\
    1 \\
    1 \\
    0
    \end{bmatrix}
\]

Así:
\[
    K_{12} = -\frac{1}{2} \left( 
        \begin{bmatrix} 1 & 0 \\ 0 & 1 \end{bmatrix} 
        - \begin{bmatrix} 1 \\ 0 \end{bmatrix} 
        \begin{bmatrix} 1 & 0 \end{bmatrix} \right) = -\frac{1}{2} \begin{bmatrix} 0 & 0 \\ 0 & 1 \end{bmatrix},
\]
\[
    K_{23} = -\frac{1}{2} \left( 
        \begin{bmatrix} 1 & 0 \\ 0 & 1 \end{bmatrix} 
        - \begin{bmatrix} \frac{1}{2} \\ \frac{\sqrt{3}}{2} \end{bmatrix} 
        \begin{bmatrix} \frac{1}{2} & \frac{\sqrt{3}}{2} \end{bmatrix} \right) 
        = -\frac{1}{2} \begin{bmatrix} \frac{3}{4} & -\frac{\sqrt{3}}{4} \\ -\frac{\sqrt{3}}{4} & \frac{1}{4} \end{bmatrix},
\]
\[
    K_{34} = -\frac{1}{2} \left( 
        \begin{bmatrix} 1 & 0 \\ 0 & 1 \end{bmatrix} 
        - \begin{bmatrix} -1 \\ 0 \end{bmatrix} 
        \begin{bmatrix} -1 & 0 \end{bmatrix} \right) = -\frac{1}{2} \begin{bmatrix} 0 & 0 \\ 0 & 1 \end{bmatrix},
\]
\[
    K_{14} = -\frac{1}{2} \left( 
        \begin{bmatrix} 1 & 0 \\ 0 & 1 \end{bmatrix} 
        - \begin{bmatrix} \frac{1}{2} \\ \frac{\sqrt{3}}{2} \end{bmatrix} 
        \begin{bmatrix} \frac{1}{2} & \frac{\sqrt{3}}{2} \end{bmatrix} \right) 
        = -\frac{1}{2} \begin{bmatrix} \frac{3}{4} & -\frac{\sqrt{3}}{4} \\ -\frac{\sqrt{3}}{4} & \frac{1}{4} \end{bmatrix}.
\]
\[
    K_{24} = -\frac{0}{2} \left( 
        \begin{bmatrix} 1 & 0 \\ 0 & 1 \end{bmatrix} 
        - \begin{bmatrix} -\frac{1}{2} \\ \frac{\sqrt{3}}{2} \end{bmatrix} 
        \begin{bmatrix} -\frac{1}{2} & \frac{\sqrt{3}}{2} \end{bmatrix} \right) 
        = \begin{bmatrix} 0 & 0 \\ 0 & 0 \end{bmatrix}.
\]
Entonces:
\[
    K_{12} = \begin{bmatrix} 0 & 0 \\ 0 & -\frac{1}{2} \end{bmatrix}, \quad
    K_{23} = \begin{bmatrix} -\frac{3}{8} & \frac{\sqrt{3}}{8} \\ \frac{\sqrt{3}}{8} & -\frac{1}{8} \end{bmatrix}, \quad
    K_{34} = \begin{bmatrix} 0 & 0 \\ 0 & -\frac{1}{2} \end{bmatrix}, \quad
    K_{14} = \begin{bmatrix} -\frac{3}{8} & \frac{\sqrt{3}}{8} \\ \frac{\sqrt{3}}{8} & -\frac{1}{8} \end{bmatrix},
\]
\[
    K_{24} = \begin{bmatrix} 0 & 0 \\ 0 & 0 \end{bmatrix}.
\]
El Hessiano Geométrico tiene la forma:
\renewcommand{\arraystretch}{1.5}
\[
    \mathcal{H}_{\mathrm{geom}} = 
    \left[
    \begin{array}{c|c|c|c}
        \underbrace{K_{12} + K_{41}}_{\text{Disco } 1} & -K_{12} & 0 & -K_{41} \\ \hline
        -K_{12} & \underbrace{K_{12} + K_{23}}_{\text{Disco } 2} & -K_{23} & 0 \\ \hline
        0 & -K_{23} & \underbrace{K_{23} + K_{34}}_{\text{Disco } 3} & -K_{34} \\ \hline
        -K_{41} & 0 & -K_{34} & \underbrace{K_{34} + K_{41}}_{\text{Disco } 4}
    \end{array}
    \right]
\]
\renewcommand{\arraystretch}{1}
Sustituyendo cada bloque, obtenemos el Hessiano geométrico:
\[
    \mathcal{H}_{\mathrm{geom}} = \begin{bmatrix}
        -\frac{3}{8} & \frac{\sqrt{3}}{8} & 0 & 0 & 0 & 0 & \frac{3}{8} & -\frac{\sqrt{3}}{8} \\[0.8em]
        \frac{\sqrt{3}}{8} & -\frac{5}{8} & 0 & \frac{1}{2} & 0 & 0 & -\frac{\sqrt{3}}{8} & \frac{1}{8} \\[0.8em]
        0 & 0 & -\frac{3}{8} & \frac{\sqrt{3}}{8} & \frac{3}{8} & -\frac{\sqrt{3}}{8} & 0 & 0 \\[0.8em]
        0 & \frac{1}{2} & \frac{\sqrt{3}}{8} & -\frac{5}{8} & -\frac{\sqrt{3}}{8} & \frac{1}{8} & 0 & 0 \\[0.8em]
        0 & 0 & \frac{3}{8} & -\frac{\sqrt{3}}{8} & -\frac{3}{8} & \frac{\sqrt{3}}{8} & 0 & 0 \\[0.8em]
        0 & 0 & -\frac{\sqrt{3}}{8} & \frac{1}{8} & \frac{\sqrt{3}}{8} & -\frac{5}{8} & 0 & \frac{1}{2} \\[0.8em]
        \frac{3}{8} & -\frac{\sqrt{3}}{8} & 0 & 0 & 0 & 0 & -\frac{3}{8} & \frac{\sqrt{3}}{8} \\[0.8em]
        -\frac{\sqrt{3}}{8} & \frac{1}{8} & 0 & 0 & 0 & \frac{1}{2} & \frac{\sqrt{3}}{8} & -\frac{5}{8}
    \end{bmatrix}
\]

Sumando ambos Hessianos, obtenemos el Hessiano Total:
\[
    \mathcal{H}_{\mathrm{total}} = \mathcal{H}_{eucl} + \mathcal{H}_{\mathrm{geom}} =
        \begin{bmatrix}
        \; 0 & 0 & 0 & 0 & 0 & 0 & 0 & 0 \; \\[0.4em]
        \; 0 & 0 & 0 & 0 & 0 & 0 & 0 & 0 \; \\[0.4em]
        \; 0 & 0 & 0 & 0 & 0 & 0 & 0 & 0 \; \\[0.4em]
        \; 0 & 0 & 0 & 0 & 0 & 0 & 0 & 0 \; \\[0.4em]
        \; 0 & 0 & 0 & 0 & 0 & 0 & 0 & 0 \; \\[0.4em]
        \; 0 & 0 & 0 & 0 & 0 & 0 & 0 & 0 \; \\[0.4em]
        \; 0 & 0 & 0 & 0 & 0 & 0 & 0 & 0 \; \\[0.4em]
        \; 0 & 0 & 0 & 0 & 0 & 0 & 0 & 0 \; \\[0.4em]
        \end{bmatrix}
\]

Para restringir a $\text{Roll}(c)$ consideramos una base $Z$:
\[
    Z = 
    \begin{bmatrix}
    \; \frac{\sqrt{3}}{2}  & 1 & -\frac{\sqrt{3}}{2}  \; \\[0.4em]
    \; -\frac{1}{2}        & 0 & \frac{3}{2}          \; \\[0.4em]
    \; \frac{\sqrt{3}}{2}  & 1 & -\frac{\sqrt{3}}{2}   \; \\[0.4em]
    \; \frac{1}{2}         & 0 & \frac{1}{2}          \; \\[0.4em]
    \; 0         & 1 & 0                    \; \\[0.4em]
    \; 1         & 0 & 0                    \; \\[0.4em]
    \; 0         & 1 & 0                    \; \\[0.4em]
    \; 0         & 0 & 1                    \; \\[0.4em]
    \end{bmatrix}
\]

Al realizar la operación obtenemos nuevamente la matriz nula pero con la dimensión reducida ($3 \times 3$):

\[
    \mathcal{H}_{\mathrm{intr}} = Z^\top_{3 \times 8} \cdot \mathcal{H}_{\mathrm{total}_{8 \times 8}} \cdot Z_{8 \times 3} = 
    \begin{bmatrix}
    \; 0 & 0 & 0 \; \\[0.4em]
    \; 0 & 0 & 0 \; \\[0.4em]
    \; 0 & 0 & 0 \; \\[0.4em]
    \end{bmatrix}
\]
El Hessiano intrínseco es la matriz nula de dimensión $3 \times 3$. Esto implica que tiene 3 autovalores 0,
\[
    \boxed{
    \text{Autovalores de } \mathcal{H}_{\mathrm{intr}}: \quad 0, 0, 0}
\]   
que corresponden a las simetrías de traslación y rotación del sistema.


\end{document}  